\section{Introduction}
Testing is an important part of producing high quality software, but its effectiveness depends on the quality of the Test Suite. Testing textbooks often recommend coverage as one of the metrics for measuring effectiveness of a test suite. \\% \cite{perry2007effective}. \\
But unfortunately, different previous work found contradictory results in different research. For example, Gligoric et al. \cite{gligoric2015guidelines} found that, effectiveness of each test suite using mutation testing has a Kendall's $\tau$ correlation from 0.452 to 0.757 when size of the suite is not considered and from 0.585 to 0.958 when considered suite size. On a different experiment, Gopinath et al. \cite{gopinath2014code} found that, coverage is correlated with effectiveness across project with for all coverage types and for both developer-generated and automatic-generated suits. They also showed that, including suite size in the regression model did not improve the results. \\
So, due to different assumption such as small or synthetic programs, confounding influence of test suite size and using impractical adequate suites, no consensus was found in code coverage as measure of effectiveness in a test suite. That's why, Inozemtseva et al \cite{inozemtseva2014coverage} proposed three research questions.
\begin{itemize}
    \item \textbf{RQ1:} Is The Size Of A Test Suite Correlated With Effectiveness?
    \item \textbf{RQ2:} Is The Coverage Of A Test Suite Correlated With Effectiveness When Suite Size Is Ignored?
    \item \textbf{RQ3:} Is The Coverage Of A Test Suite Correlated With Effectiveness When Suite Size Is Fixed?
\end{itemize}


\begin{table}
\begin{center}
\caption{Important characteristics of the subject programs.}
 \begin{tabular}{p{2.5cm} p{.8cm} p{.8cm} p{1.1cm} p{.8cm} p{.7cm}}
 \hline
 \textbf{Property} & \textbf{Apache POI} & \textbf{Closure} & \textbf{HSQLDB} & \textbf{JFree Chart} & \textbf{Joda Time}\\[0.5ex] 
 \hline
 Total Java SLOC & 283,845 & 724,089 & 178,018 & 125,659 & 80,462 \\ 

 Test SLOC & 68,832 & 93,528 & 18,425 & 44,297 & 51,444 \\
 
 Statement coverage & 67\% & 76\% & 27\% & 54\% & 91\%\\
 
 Decision coverage & 60\% & 77\% & 17\% & 45\% & 82\%\\
 
 MC coverage & 49\% & 67\% & 9\% & 27\% & 70\%\\ [1ex] 
 \hline
 \# mutants & 27,565 & 30,779 & 50,302 & 29,699 & 9,552\\
 \# detected mutants & 17,835 & 27,325 & 50,125 & 23,585 & 8,483\\
 Equivalent mutants & 35\% & 11\% & 0.4\% & 21\% & 11\%\\[1ex]
 \hline
\end{tabular}
\end{center}
\end{table}
